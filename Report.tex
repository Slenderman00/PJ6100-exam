% !TeX root = Report.tex
\documentclass[12pt]{article}

% Package imports (organized and deduplicated)
\usepackage{biblatex}
\usepackage{changepage}
\usepackage{color}
\usepackage{enumitem}
\usepackage{float}
\usepackage{graphicx}
\usepackage{listings}
\usepackage{sectsty}
\usepackage{xcolor}
\usepackage[breaklinks=true]{hyperref}
\usepackage{xurl}
\usepackage{tikz}
\usetikzlibrary{shapes.geometric,positioning,fit,backgrounds}
\usepackage{./timing-diagrams}
\usetikzlibrary{calc}
\setcounter{biburlnumpenalty}{100}
\setcounter{biburlucpenalty}{100}
\setcounter{biburllcpenalty}{100}

\definecolor{darkblue}{RGB}{0, 0, 102} 
\hypersetup{
    colorlinks=true,
    pdfborder={0 0 0},
    linkbordercolor=white,
    urlcolor=darkblue,
    linkcolor=darkblue,
    citecolor=darkblue,
    filecolor=darkblue
}

% Make bibliography ragged right instead of justified
\AtBeginDocument{
  \renewcommand{\bibsetup}{\raggedright}
}
% Document configuration
\restylefloat{table}
\graphicspath{{./images/}}
\addbibresource{Library.bib}
\subsectionfont{\fontsize{12}{14}\selectfont}

% Author information
\author{
    Joar Heimonen\\
    \texttt{contact@joar.me}
}

% Title configuration
\title{
    \textbf{PEAK}\\[0.5em]
    \large \textbf{P}roxy \textbf{E}liminating \textbf{A}rchitecture using \textbf{K}ubernetes\\[0.3em]
    \large A research propsal
}
\date{\today}

\newcommand{\license}{
    \vspace{1em}
    \noindent\small{© 2024 Joar Heimonen\\
    This work is licensed under a \href{https://creativecommons.org/licenses/by-sa/4.0/}{Creative Commons Attribution-Sharealike 4.0 International License}.}
    \vspace{1em}
}

\begin{document}
\maketitle

\pagebreak

\tableofcontents

\pagebreak


\section{Part 1}
\subsection{Introduction}
With the adoption of IPv6 our networking landscape is about to change. Last year we proposed the architecture \textbf{PEAK} \cite{heimonenPreprintPEAKProxy} which leverages the new capabilities of IPv6
to create distributed systems that are not reliant on proxies.
Our last paper introduced a reference implementation of the PEAK architecture, while this is useful it does not allow us to benchmark and compare the performance of 
the PEAK architecture to traditional systems.
\\
\\
In this paper we propose a set of reference implementations of the PEAK architecture that will be benchmarked against a set of traditional systems.
We will also propose a set of metrics that will be used to compare the performance of the systems.
With the aim of showing that the PEAK architecture not only is viable but matches or exceeds the performance of traditional systems.

\subsection{Background}
Since the start of this century, the internet has grown from a small network of universities to a global network that connects billions of devices. Due to our limited
address space provided by IPv4 we have not been able to give every device a unique address. This has led to the development of two technologies. Network address translation (NAT) and proxies.
\\
\\
NAT allows for the creation of subnets that can be used to connect multiple devices to the internet through a single shared IP address. This is a great solution for small networks
but leads to an increase in complexity as the network grows. In a subnet consisting of servers none of these servers can be reached from the internet without the use of a proxy.
The proxy simply redirects traffic to the correct in the subnet based on arbitrary rules.
\\
\\
While proxies works fine they introduce a single point of failure in distributed systems. If the proxy stops working, all the servers within the subnet become unreachable.



\addcontentsline{toc}{section}{References}
\printbibliography
\license
\end{document}